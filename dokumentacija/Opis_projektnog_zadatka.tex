\chapter{Opis projektnog zadatka}
		
		
		\textit{Ovaj projekt će se baviti osmišljanjem online galerije. Prilikom kupnje ili pregleda slika , zainteresirani ljudi najćešće moraju obići nekoliko različlitih galerija te pogledati nekoliko  izložbi. Također , najčešće ne mogu odmah kupiti sliku te sama kupnja nije baš jednostavna , pogotovo ako smo zainteresirani za sliku koja je na izložbi. 
			 U vrijeme pisanja ovog  teksta , na snazi su mjere za suzbijanje virusa COVID-19 pa je ograničen broj ljudi u zatvorenim prostorima te je preporučeno držanje socijalne distance . Galerije su definitivno  mjesta gdje se lako može skupiti  veći broj ljudi na manjoj površini pa je online galerija ne samo praktična zamjena pravih galerija  već i puno sigurnija opcija. 	
		\vspace{10mm} 
}
		\graphicspath{ {./slike/} }
				\begin{figure}[H]

					\includegraphics[width=\textwidth,height=\textheight,keepaspectratio]{onlinegallery_art}
					\caption{\newline Snimka zaslona je slikana 7.11.2020. sa  stranice : https://onlinegallery.art/en/ }

				\end{figure}
		\graphicspath{ {./slike/} }
				\begin{figure}[H]

					\includegraphics[width=\textwidth,height=\textheight,keepaspectratio]{saatchiart_com}
					\caption{\newline Snimka zaslona je slikana 7.11.2020. sa  stranice : https://www.saatchiart.com/}

				\end{figure}	
		
		\vspace{10mm} 
			
		
		\textit{Za razliku od sličnih stranica ovdje navedenih , ovaj projekt će biti baziran na prezentaciji izložbi u kojima se nalaze djela , a ne na prezentaciji samih djela. 
			Također će projekt 	imati i komentare na pojedina djela što nije podržano na navedenim web stranicama. O samoj implementaciji stranica ne možemo govoriti bez da znamo koje su funkcionalnosti  dane administratoru.
			\newline Online galerija namijenjena je široj publici ljubitelja likovnih djela , neovisno jesu li kupci istih ili samo posjetitelji izložbe.
			\vspace{3mm}
			\newline Web stranici se pristupa sa jednom od četiri uloge ; neregistrirani korisnik , umjetnik, registrirani posjetitelj ili administrator. Neregistrirani korisnik ima najmanje ovlasti te on može  sortirati izložbe po određenim svojstvima te može vidjeti pregled aktivnih i nadolazećih izložba. 
			\newline Na početnoj stranici se nalazi popis izložbi. Odabirom kategorije po kojoj se želi sortirati , djela se sortiraju.
			\newline Za registraciju u sustav korisniku su potrebni ime , prezime , email i PayPal račun.
			\vspace{3mm}
			\newline Registracijom u sustav neregistrirani korisnik postaje registirani posjetitelj te dobiva ovlasti za ulazak u izložbu te pregled ,komentiranje i kupnju djela. On također može  pregledavati vlastite izvršene transakcije.
			\newline Nakon ulaska u početnu stranicu , klikom na izložbu ulazi se u istu. Otvara se stranica izložbe te se u njemu nalaze sva djela te izložbe.
			\newline Prelaskom miša preko slike djela , pojavljuje se slika košarica ,a njenim odabirom slika je dodana u košaricu. Ikona košarice nalazi se u gornjem desnom kutu te se klikom na nju  otvaraju sva djela koja su odabrana za kupnju. Kupnja se zatim potvrđuje te se korisnik prijavljuje na Paypal račun te uplaćuje novac. 
			\newline Klikom na djelo otvara se nova stranica djela u kojem je moguće komentirati. 
			\vspace{3mm}
			\newline Umjetnik prilikom registracije prilaže i svoj portfolio u .pdf formatu. On ima sve ovlasti registriranog posjetitelja , ali on može i dodavati,pregledavati i uređivati kolekcije radova na  svojem računu te prijaviti jednu ili više svojih kolekcija na natječaj.
			\vspace{3mm}
			\newline Administrator je uloga s najvišim ovlastima te on može stvarati natječaje , odlučivati o prijavi kolekcije na natječaj , stvarati izložbu , pregledavati sve izvršene transakcije ,  pregledavati i uklanjati sve račune , zatvarati izložbe te brisati komentare.
			\vspace{3mm}
			\newline Iako je web stranica namjenjena galeriji , uz minimalne modifikacije stranica bi mogla služiti online dućan. U tom slučaju korisnici bi otvarali kategorije artikla primjerice kućanske  potrepštine , hranu , piće itd... Nakon otvaranja kategorije  pojavljuju se artikli koje je moguće dodati u košaricu. Komentari na artiklima bi bili mogući u smislu recenzije proizvoda.	
			Jedine promjene na stranici bi trebale biti promjene naziva jer su sve ostale funkcije već implementirane. 
			\vspace{3mm}
			\newline U budućnosti bi se na stranici moglo dodati više različitih opcija plaćanja poput plaćanja bankovnim i kreditnim karticama te bi bilo korisno podržati mobilne korisnike razvojem  mobilne stranice.
}
		
