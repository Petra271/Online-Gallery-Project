\chapter{Dnevnik promjena dokumentacije}

		\begin{longtabu} to \textwidth {|X[2, l]|X[13, l]|X[3, l]|X[3, l]|}
			\hline \multicolumn{1}{|l|}{\textbf{Rev.}}	& \multicolumn{1}{l|}{\textbf{Opis promjene/dodatka}} & \multicolumn{1}{|l|}{\textbf{Autori}} & \multicolumn{1}{l|}{\textbf{Datum}} \\[3pt] \hline
			\endfirsthead
			
			\hline \multicolumn{1}{|l|}{\textbf{Rev.}}	& \multicolumn{1}{l|}{\textbf{Opis promjene/dodatka}} & \multicolumn{1}{|l|}{\textbf{Autori}} & \multicolumn{1}{l|}{\textbf{Datum}} \\[3pt] \hline
			\endhead
			
			\hline 
			\endlastfoot
			
			0.1 & Napravljen predložak, dodan prvi dio opisa obrazaca uporabe.	& Kurtović & 21.10.2020. 		\\[3pt] \hline
			0.2 & Dodani aktori i njihovi funkcionalni zahtjevi.	& Novinc & 21.10.2020. 		\\[3pt] \hline  
			0.3 & Dodan drugi dio opisa obrazaca uporabe.	& Blagaić & 21.10.2020. 		\\[3pt] \hline
			0.4 & Dodan treći dio opisa obrazaca uporabe.	& Mahović & 21.10.2020. 		\\[3pt] \hline  
			0.5 & Dodan četvrti dio opisa obrazaca uporabe.	& Šestak & 21.10.2020. 		\\[3pt] \hline
			0.5.1 & Izmjene u obrascima uporabe.	& Šestak & 21.10.2020. 		\\[3pt] \hline   
			0.6 & Dodan peti dio opisa obrazaca uporabe.	& Ilić & 21.10.2020. 		\\[3pt] \hline
			0.7 & Dodan šesti dio opisa obrazaca uporabe.	& Zubčić & 21.10.2020. 		\\[3pt] \hline
			0.7.1 & Ispravljene greške u obrascima uporabe.	& Zubčić & 21.10.2020. 		\\[3pt] \hline
			0.8 & Dodani dijagrami obrazaca uporabe i napravljene izmjene obrazaca.	& Ilić & 04.11.2020. 		\\[3pt] \hline
			0.9 & Dodan opis projektnog zadatka. & Šestak & 06.11.2020. 		\\[3pt] \hline
			0.9.1 & Dodane slike u opis projektnog zadatka i ispravljanje pogrešaka. & Šestak & 07.11.2020. 		\\[3pt] \hline
			0.10 & Dodani sekvencijski dijagrami i ostali zahtjevi. & Ilić & 08.11.2020. 		\\[3pt] \hline
			0.11 &Font i tipfeleri  & Šestak & 08.11.2020 \\[3pt] \hline
			
			          
			
			
			
		\end{longtabu}
	
	
		\textit{Moraju postojati glavne revizije dokumenata 1.0 i 2.0 na kraju prvog i drugog ciklusa. Između tih revizija mogu postojati manje revizije već prema tome kako se dokument bude nadopunjavao. Očekuje se da nakon svake značajnije promjene (dodatka, izmjene, uklanjanja dijelova teksta i popratnih grafičkih sadržaja) dokumenta se to zabilježi kao revizija. Npr., revizije unutar prvog ciklusa će imati oznake 0.1, 0.2, …, 0.9, 0.10, 0.11.. sve do konačne revizije prvog ciklusa 1.0. U drugom ciklusu se nastavlja s revizijama 1.1, 1.2, itd.}