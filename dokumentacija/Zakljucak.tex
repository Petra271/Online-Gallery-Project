\chapter{Zaključak i budući rad}
		

		{Zadatak naše grupe bio je razvoj online galerije. Na stranici posjetitelji mogu pregledavati izložbe , ostavljati komentare te kupovati djela , dok umjetnici mogu svoja djela dodavati na izložbu te ista prodavati. Zadatak je uspješno ostvaren te je sam proces podijeljen u dvije faze. 
		U prvoj fazi je okupljen tim te se pretežito bavilo dokumentacijom projekta. Zbog dobre , potpune i cjelovite dokumentacije, ulazak u drugu fazu je sa potpuno jasnim ciljevima i dobro definiranim zahtjevima. Pomoć u tome su  obrasci uporabe , model baze podataka , sekvencijski dijagrami te
		model razreda koji su napravljeni tijekom prve faze. Druga faza počela je s proučavanjem alata ,sučelja i programskih jezika koji će biti korišteni u razvoju aplikacije. Odmah na početku je došlo do podjele u timove za frontend i backend  koji su počeli sa izradom. Osim same izrade aplikacije , u drugoj fazi 
		također je  dokumentiran projekt i to dijagramima stanja , aktivnosti , komponenti i razmještaja. Komunikacija unutar tima se odvijala pretežito preko WhatsAppa dok su sastanci te komunikacija frontend i backend timova bili preko Discorda. 
		Proširenje aplikacije bilo bi prvenstveno razvojem mobilne aplikacije čime bi aplikacija dobila još više korisnika. Također moguće je i dodavanje novih načina plaćanja te razvoj stranice za dostavu u kojoj bi korisnik mogao birati način preuzimanja slike odnosno dostave iste.
		 Ovaj projekt dao je svim članovima tima iskustvo ne samo u rukovanju s programima poput LaTexa  , Asteha , radu s programskim jezicima HTML,CSS,JS te radu s radnim okvirom Vue.js
		već i iskustvo timskog rada na dužem i kompleksnijem projektu što će im sigurno biti korisno u budućim projektima tijekom školovanja te unutar radne zajednice na budućim poslovima.
		Za daljni razvoj aplikacije najvažnije bi bilo razraditi dokumentaciju poput one u prvoj fazi projekta uz koju je puno lakša i jednostavnija izrada bilo kakve nadogradnje aplikacije.
		 }
		
		\eject 