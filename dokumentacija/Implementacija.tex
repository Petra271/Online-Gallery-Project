\chapter{Implementacija i korisničko sučelje}
		
		
		\section{Korištene tehnologije i alati}
		
			{Komunikacija u timu realizirana je korištenjem aplikacija \underline{WhatsApp}\footnote{\url{https://www.whatsapp.com/}} i \underline{Discord}\footnote{\url{https://discord.com/}}. Za izradu UML dijagrama korišten je alat \underline{Astah UML}\footnote{\url{https://astah.net/products/astah-uml/}}, a kao sustav za upravljanje izvornim kodom \underline{Git}\footnote{\url{https://git-scm.com/}}. Udaljeni repozitorij projekta je dostupan na web platformi \underline{GitLab}\footnote{\url{https://gitlab.com/}}.
				
			Kao razvojno okruženje za \textit{frontend} korišten je \underline{Microsoft Visual Studio}\footnote{\url{https://visualstudio.microsoft.com/}} - integrirano je razvojno okruženje (IDE) tvrtke Microsoft. Prvenstveno se koristi za razvoj računalnih programa za operacijski sustav Windows, kao i za web-stranice, web-aplikacije, web-usluge i mobilne aplikacije. Visual Studio za razvoj softvera koristi razne Microsoftove platforme.
			
			Kao razvojno okruženje za \textit{backend} korišten je \underline{IntelliJ IDEA}\footnote{\url{https://www.jetbrains.com/idea/}} - integrirano je razvojno okruženje (IDE) tvrke JetBrains. Prvenstveno se koristi za razvoj programa u programskom jeziku Java, ali podržava i druge jezike. Dostupno je u dvije verzije: besplatna Community Edition i komercijalna Ultimate Edition.
			
			Aplikacije je napisana koristeći radni okvir \underline{Spring Boot}\footnote{\url{https://spring.io/projects/spring-boot}} i jezik \underline{Java}\footnote{\url{https://www.java.com/}} za izradu \textit{backenda} te \underline{Vue.js}\footnote{\url{https://vuejs.org/}} i jezik \underline{Javascript}\footnote{\url{https://www.javascript.com/}} za izradu \textit{frontenda}. Vue.js je open-source MVC \textit{frontend} radni okvir pisan u Javascriptu za izgradnju korisničkih sučelja i jednostraničnih aplikacija gdje se glavni dokument dinamički
			mijenja. Radni okvir Spring Boot je specijalizacija radnog okvira \underline{Spring}\footnote{\url{https://spring.io/projects/spring-framework}} s ciljem jednostavnijeg i bržeg oblikovanja web aplikacija. U svojoj automatskoj konfiguraciji olakšava posao programeru jer više toga uobičajenog u web aplikacijama već ima podešeno.
			
			Za bazu podataka koristili smo \underline{PostgreSQL}\footnote{\url{https://www.postgresql.org/}}.
			
			\eject 
		
	
		\section{Ispitivanje programskog rješenja}
			
			\subsection{Ispitivanje komponenti}

			{Sljedeći jedinični testovi su korišteni za ispitivanje jedinica nad razredima koji implementiraju neke od temeljnih funkcionalnosti.}
				\begin{figure}[H]
				
				\includegraphics[width=\textwidth,height=\textheight,keepaspectratio]{registration_test}
				\caption{Jedinični testovi koji testiraju registraciju novih posjetitelja i novih umjetnika}
				
				\end{figure}
			
				\begin{figure}[H]
					
					\includegraphics[width=\textwidth,height=\textheight,keepaspectratio]{login_test}
					\caption{Jedinični testovi koji ispituju prijavu posjetitelja i umjetnika}
					
				\end{figure}
			
				\begin{figure}[H]
				
				\includegraphics[width=\textwidth,height=\textheight,keepaspectratio]{registration_with_same_email_test}
				\caption{Jedinični test koji ispituje da se vraća prikladni kod za grešku kada se pri registraciji koristi e-mail postojećeg korisnika}
				
				\end{figure}
			
				\begin{figure}[H]
					
					\includegraphics[width=\textwidth,height=\textheight,keepaspectratio]{post_artwork_test}
					\caption{Jedinični test koji ispituje dodavanje nove slike u kolekciju umjetnika}
					
				\end{figure}
			
				\begin{figure}[H]
					
					\includegraphics[width=\textwidth,height=\textheight,keepaspectratio]{create_collection_test}
					\caption{Jedinični test koji ispituje stvaranje nove umjetnikove kolekcije}
					
				\end{figure}
			
			{Sljedeće slike prikazuju rezultate pokretanja jediničnih testova. Zbog načina na koji su pisani nisu mogli biti pokrenuti svi istovremeno.}
				\begin{figure}[H]
					
					\includegraphics[width=\textwidth,height=\textheight,keepaspectratio]{test_results1}
					\caption{Rezultati prvih 5 jediničnih testova, zadnja dva nisu bili pokretani}
					
				\end{figure}
			
				\begin{figure}[H]
					
					\includegraphics[width=\textwidth,height=\textheight,keepaspectratio]{test_results2}
					\caption{Rezultat jediničnog testa koji ispituje dodavanje slike u kolekciju umjetnika}
					
				\end{figure}
				
				\begin{figure}[H]
					
					\includegraphics[width=\textwidth,height=\textheight,keepaspectratio]{test_results3}
					\caption{Rezultat jediničnog testa koji ispituje stvaranje nove kolekcije}
					
				\end{figure}
			
			\eject	
			
			\subsection{Ispitivanje sustava}
			
			 \begin{figure}[H]
			 	
			 	\includegraphics[width=\textwidth,height=\textheight,keepaspectratio]{3}
			 	\caption{}
			 	
			 \end{figure}
		 
		  \begin{figure}[H]
		 	
		 	\includegraphics[width=\textwidth,height=\textheight,keepaspectratio]{4}
		 	\caption{}
		 	
		 \end{figure}
	 
	  \begin{figure}[H]
	 	
	 	\includegraphics[width=\textwidth,height=\textheight,keepaspectratio]{5}
	 	\caption{}
	 	
	 \end{figure}
 
  \begin{figure}[H]
 	
 	\includegraphics[width=\textwidth,height=\textheight,keepaspectratio]{6}
 	\caption{}
 	
 \end{figure}
 \begin{figure}[H]
	
	\includegraphics[width=\textwidth,height=\textheight,keepaspectratio]{1}
	\caption{}
	
\end{figure}

 \begin{figure}[H]
	
	\includegraphics[width=\textwidth,height=\textheight,keepaspectratio]{2}
	\caption{}
	
\end{figure}
		
		
		\section{Dijagram razmještaja}
			
			{Slika prikazuje specifikacijski dijagram razmještaja koji opisuje topologiju sustava. Sustav se sastoji od poslužiteljskog računala na kojem se nalaze web poslužitelj i poslužitelj baze podataka. Web aplikacija ovisna je o bazi podataka. Klijenti koriste web preglednik za pristup aplikaciji, a komunikacija klijent – poslužitelj odvija se preko HTTP protokola.}
			 
			 \begin{figure}[H]
			 	
			 	\includegraphics[width=\textwidth,height=\textheight,keepaspectratio]{dijagram_razmjestaja}
			 	\caption{Specifikacijski dijagram razmještaja}
			 	
			 \end{figure}
			
			\eject 
		
		\section{Upute za puštanje u pogon}
		
			\subsubsection{Instalacija poslužitelja baze podataka}
		
			Prije instalacije potrebno je preuzeti instalacijski program sa sljedeće poveznice:
				\href{https://www.postgresql.org/download/}{\underline{https://www.postgresql.org/download/}}.
			\newline\newline
			Napomene za instalaciju:
			\begin{list}{$\circ$}{}
				\item  kod odabira komponenti sustava koje će biti
				instalirane, obavezno je označiti komponente
	 			\textbf{PostgreSQL Server} i \textbf{PgAdmin} 

				\item  ostaviti predložena vrata za pristup
				sustavu baze podataka: \textbf{5432}
				\item  ostale postavke, poput lozinke postgres korisnika, postaviti proizvoljno
			\end{list}
		
			\subsubsection{Instalacija Mavena}
			Ukoliko Maven već nije instaliran, potrebno ga je preuzeti s poveznice\newline
			\href{https://maven.apache.org/download.cgi}{\underline{https://maven.apache.org/download.cgi}} i instalirati.
		
			\subsubsection{Korištenje pgAdmin programa za pristup PostgreSQL sustavu baze podataka}
			
			Pri prvom pokretanju programa pgAdmin potrebno je upisati
			„master“ lozinku – istu onu koju je postavljena za postgres korisnika.
			Nakon unosa lozinke otvoriti u prozoru „Browser“ listu
			„Servers“ i kliknuti na znak pored „PostgreSQL 12“.
			Ponovno unijeti lozinku za postgres korisnika.
			
			\subsubsection{Konfiguracija veze s bazom podataka}
			U datoteci \textbf{application.properties} koja se nalazi u direktoriju
			\newline
			„backend/src/main/resources“, potrebno je postaviti postgres korisnika i lozinku koja je stavljena tijekom instalacije poslužitelja baze podataka. Pri prvom pokretanju aplikacije, baza se automatski stvara.
			
			\begin{figure}[H]
				\includegraphics[width=\textwidth,height=\textheight,keepaspectratio]{db_username_pass}
			\end{figure}
		
			\eject
		
			\subsubsection{Pokretanje aplikacije}

			U terminalu se pozicionirati u direktorij  „backend“. Izvesti naredbu:
			\begin{list}{$\circ$}{}
				\item  mvn spring-boot:run -Drun.arguments=-{}-server.port=8080
			\end{list}
			Ova naredba pokrenut će Spring Boot aplikaciju na portu 8080. Broj porta može se promijeniti, ali onda je potrebno u datoteci \textbf{.env} u direktoriju „frontend“ promijeniti broj porta na koji se šalju zahtjevi.
			
			\begin{figure}[H]
				\includegraphics[width=\textwidth,height=\textheight,keepaspectratio]{change_port}
			\end{figure}
		
		\noindent Pozicionirati u direktorij „frontend“. Izvesti naredbe:
		\begin{list}{$\circ$}{}
			\item  npm install
			\item  npm run serve -{}- -{}-port 8081
		\end{list}
		Prva naredba instalirat će sve potrebne pakete, a druga će pokrenuti Vue JS aplikaciju na portu 8081.
		