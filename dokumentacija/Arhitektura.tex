\chapter{Arhitektura i dizajn sustava}
		
		\textbf{\textit{dio 1. revizije}}\\

		\textit{ Potrebno je opisati stil arhitekture te identificirati: podsustave, preslikavanje na radnu platformu, spremišta podataka, mrežne protokole, globalni upravljački tok i sklopovsko-programske zahtjeve. Po točkama razraditi i popratiti odgovarajućim skicama:}
	\begin{itemize}
		\item 	\textit{izbor arhitekture temeljem principa oblikovanja pokazanih na predavanjima (objasniti zašto ste baš odabrali takvu arhitekturu)}
		\item 	\textit{organizaciju sustava s najviše razine apstrakcije (npr. klijent-poslužitelj, baza podataka, datotečni sustav, grafičko sučelje)}
		\item 	\textit{organizaciju aplikacije (npr. slojevi frontend i backend, MVC arhitektura) }		
	\end{itemize}

	
		

		

				
		\section{Baza podataka}
			
			\textbf{\textit{dio 1. revizije}}\\
			
		\textit{Potrebno je opisati koju vrstu i implementaciju baze podataka ste odabrali, glavne komponente od kojih se sastoji i slično.}
		
			\subsection{Opis tablica}
			

				
				
				\begin{longtabu} to \textwidth {|X[10, l]|X[6, l]|X[14, l]|}
					
					\hline \multicolumn{3}{|c|}{\textbf{Korisnik}}	 \\[3pt] \hline
					\endfirsthead
					
					\hline \multicolumn{3}{|c|}{\textbf{Korisnik}}	 \\[3pt] \hline
					\endhead
					
					\hline 
					\endlastfoot
					
					\cellcolor{LightGreen}Id (PK) & INT&  Primarni ključ za korisnika	\\ \hline
					KorisnickoIme& VARCHAR &  Korisničko ime korisnika, ujedno i sekundarni ključ jer je jedinstveno  za svakog korisnika\\ \hline 
					Ime	& VARCHAR &   Ime korisnika\\ \hline 
					Prezime & VARCHAR&   Prezime korisnika		\\ \hline 
					Email & VARCHAR &  E-mail korisnika ,ujedno i sekundarni ključ jer je e-mail jedinstven za svakog korisnika  \\ \hline 
					Lozinka & VARCHAR&  Lozinka korisnika	\\ \hline 
					PayPalEmail & VARCHAR&  Lozinka korisnika	\\ \hline 
					RazinaOvlasti & INT&  Oznaka razine ovlasti korisnika	\\ \hline 
					PdfPortfolio & BYTEA&  Portfolio umjetnika u obliku pdfa , samo za umjetnike	\\ \hline 
					
					
				\end{longtabu}

				
				\begin{longtabu} to \textwidth {|X[10, l]|X[6, l]|X[14, l]|}
					
					\hline \multicolumn{3}{|c|}{\textbf{Kolekcija}}	 \\[3pt] \hline
					\endfirsthead
					
					\hline \multicolumn{3}{|c|}{\textbf{Kolekcija}}	 \\[3pt] \hline
					\endhead
					
					\hline 
					\endlastfoot
					
					\cellcolor{LightGreen}Id(PK) & INT&   Primarni ključ za kolekciju	\\ \hline
					Naziv	& VARCHAR &   Naziv kolekcije	\\ \hline 
					Opis & VARCHAR &  Opis kolekcije  \\ \hline 
					Stil & VARCHAR&  Stil kolekcije	\\ \hline 
					\cellcolor{LightBlue} IdUmjetnik (FK)& INT &  Strani ključ vezan uz korisnika s ulogom umjetnik	\\ \hline 
					
					
				\end{longtabu}

				\begin{longtabu} to \textwidth {|X[10, l]|X[6, l]|X[14, l]|}
					
					\hline \multicolumn{3}{|c|}{\textbf{Komentar}}	 \\[3pt] \hline
					\endfirsthead
					
					\hline \multicolumn{3}{|c|}{\textbf{Komentar}}	 \\[3pt] \hline
					\endhead
					
					\hline 
					\endlastfoot
					
					\cellcolor{LightGreen}Id (PK)& INT&  Primarni ključ komentara	\\ \hline
					Tekst	& VARCHAR &   Tekst komentara	\\ \hline 
					Vrijeme & TIMESTSAMP &  Vrijeme kada je korisnik ostavio komentar  \\ \hline 
					\cellcolor{LightBlue}IdDjelo(FK) & INT&  Strani ključ vezan uz djelo	\\ \hline 
					\cellcolor{LightBlue} IdAutor(FK)& INT &  Strani ključ vezan uz korisnika koji je komentirao	\\ \hline 
					
					
				\end{longtabu}

				\begin{longtabu} to \textwidth {|X[10, l]|X[6, l]|X[14, l]|}
					
					\hline \multicolumn{3}{|c|}{\textbf{Transakcija}}	 \\[3pt] \hline
					\endfirsthead
					
					\hline \multicolumn{3}{|c|}{\textbf{Transakcija}}	 \\[3pt] \hline
					\endhead
					
					\hline 
					\endlastfoot
					
					\cellcolor{LightGreen}Id(PK) & INT& Primarni ključ transakcije  	\\ \hline
					Iznos	& FLOAT &   	Iznos plaćene cijene u kunama\\ \hline 
					Vrijeme & TIMESTAMP &   Vrijeme kada je transakcija tj. uplata sredstava završena\\ \hline 
					\cellcolor{LightBlue}IdDjelo(FK) & VARCHAR	&  Strani ključevi vezani uz  kupljena djela razmaknuti zarezom	\\ \hline 
					\cellcolor{LightBlue}IdPlatitelj(FK) & INT	&  Strani ključ vezan uz korisnika koji je uplatio sredstva\\ \hline 
					\cellcolor{LightBlue} IdPrimatelj(FK)& INT &   Strani ključ vezan uz autora djela tj. umjetnika koji je naslikao djelo	\\ \hline 
					
					
				\end{longtabu}

				\begin{longtabu} to \textwidth {|X[10, l]|X[6, l]|X[14, l]|}
					
					\hline \multicolumn{3}{|c|}{\textbf{KolekcijaNatjecaj}}	 \\[3pt] \hline
					\endfirsthead
					
					\hline \multicolumn{3}{|c|}{\textbf{KolekcijaNatjecaj}}	 \\[3pt] \hline
					\endhead
					
					\hline 
					\endlastfoot
					
					 IdKolekcija(PK,FK)	& INT &   Dio primarnog ključa vezan uz kolekciju,ujedno i  strani ključ	\\ \hline 
					 IdNatjecaj(PK,FK)	& INT &  Dio primarnog ključa vezan uz natječaj, ujedno i  strani ključ	\\ \hline 
					
					
				\end{longtabu}

				\begin{longtabu} to \textwidth {|X[10, l]|X[6, l]|X[14, l]|}
					
					\hline \multicolumn{3}{|c|}{\textbf{KolekcijaIzlozba}}	 \\[3pt] \hline
					\endfirsthead
					
					\hline \multicolumn{3}{|c|}{\textbf{KolekcijaIzlozba}}	 \\[3pt] \hline
					\endhead
					
					\hline 
					\endlastfoot
					
					 IdKolekcija(PK,FK)	& INT &   Dio primarnog ključa vezan uz kolekciju, ujedno i  strani ključ	\\ \hline 
					 IdIzlozba(PK,FK)	& INT &   Dio primarnog ključa vezan uz izložbu, ujedno i  strani ključ	\\ \hline 
					
					
				\end{longtabu}
			
				\begin{longtabu} to \textwidth {|X[10, l]|X[6, l]|X[14, l]|}
					
					\hline \multicolumn{3}{|c|}{\textbf{PosjetiteljIzlozba}}	 \\[3pt] \hline
					\endfirsthead
					
					\hline \multicolumn{3}{|c|}{\textbf{PosjetiteljIzlozba}}	 \\[3pt] \hline
					\endhead
					
					\hline 
					\endlastfoot
					
					 IdIzlozba(PK,FK)	& INT &   Dio primarnog ključa vezan uz izložbu,ujedno i  strani ključ	\\ \hline 
					 IdPosjetitelj(PK,FK)& INT &  Dio primarnog ključa vezan uz korisnika koji je posjetio izložbu , ujedno i strani ključ	\\ \hline 
					
					
				\end{longtabu}

				\begin{longtabu} to \textwidth {|X[10, l]|X[6, l]|X[14, l]|}
					
					\hline \multicolumn{3}{|c|}{\textbf{Djelo}}	 \\[3pt] \hline
					\endfirsthead
					
					\hline \multicolumn{3}{|c|}{\textbf{Djelo}}	 \\[3pt] \hline
					\endhead
					
					\hline 
					\endlastfoot
					
					\cellcolor{LightGreen}Id(PK) & INT& Primarni ključ djela 	\\ \hline
					Naziv	& VARCHAR&   Naziv djela\\ \hline 
					Opis	& VARCHAR &   Autorov tj. umjetnikov opis djela\\ \hline 
					Cijena	& FLOAT &   	Cijena djela u kunama\\ \hline 
					Stil & VARCHAR & Stil djela\\ \hline 
					blob & BYTEA & Djelo u pdf formatu \\ \hline 
					\cellcolor{LightBlue} IdKolekcija(FK)& INT &   Strani ključ vezan uz kolekciju u kojem je djelo	\\ \hline 
					
					
				\end{longtabu}
				
				\begin{longtabu} to \textwidth {|X[10, l]|X[6, l]|X[14, l]|}
					
					\hline \multicolumn{3}{|c|}{\textbf{Izlozba}}	 \\[3pt] \hline
					\endfirsthead
					
					\hline \multicolumn{3}{|c|}{\textbf{Izlozba}}	 \\[3pt] \hline
					\endhead
					
					\hline 
					\endlastfoot
					
					\cellcolor{LightGreen}Id(PK) & INT& Primarni ključ izložbe 	\\ \hline
					Naziv	& VARCHAR&   Naziv djela\\ \hline 
					Opis	& VARCHAR &   Autorov tj. umjetnikov opis djela\\ \hline 
					VrijemePocetka& TIMESTAMP &   	Vrijeme u kojem počinje izložba\\ \hline 
					VrijemeTrajanja& INTERVAL &   	Interval u kojem izložba traje\\ \hline 
					Provizija &FLOAT & Broj izražen u postocima o kojem ovisi koliko stranica uzima proviziju od cijene djela
 tj. koliko će od ukupne cijene djela novaca dobiti korisnik. Korisnik dobiva (100-Provizija)/100*CijenaDjela kuna dok ostatak novaca ide stranici.\\ \hline 
					Stil & VARCHAR & Stil izložbe\\ \hline 
					\cellcolor{LightBlue} IdNatjecaj(FK)& INT &   Strani ključ vezan uz natječaj u kojem je djelo	\\ \hline 
					
					
				\end{longtabu}

				\begin{longtabu} to \textwidth {|X[10, l]|X[6, l]|X[14, l]|}
					
					\hline \multicolumn{3}{|c|}{\textbf{Natjecaj}}	 \\[3pt] \hline
					\endfirsthead
					
					\hline \multicolumn{3}{|c|}{\textbf{Natjecaj}}	 \\[3pt] \hline
					\endhead
					
					\hline 
					\endlastfoot
					
					\cellcolor{LightGreen}Id(PK) & INT& Primarni ključ natječaja 	\\ \hline
					Naziv	& VARCHAR&   Naziv natječaja\\ \hline 
					Opis	& VARCHAR &   Administratorov opis natječaja\\ \hline 
					VrijemePocetka& TIMESTAMP &   	Vrijeme u kojem počinje natječaj\\ \hline 
					VrijemeTrajanja& INTERVAL &   	Interval u kojem natječaj traje\\ \hline 
					Provizija &FLOAT & Broj izražen u postocima o kojem ovisi koliko stranica uzima proviziju od cijene djela
 tj. koliko će od ukupne cijene djela novaca dobiti korisnik. Korisnik dobiva (100-Provizija)/100*CijenaDjela kuna dok ostatak novaca ide stranici.\\ \hline 
					Stil & VARCHAR & Stil natječaja\\ \hline 
					
					
				\end{longtabu}

			
			\subsection{Dijagram baze podataka}
				\begin{figure}[H]
					
					\includegraphics[width=\textwidth,height=\textheight,keepaspectratio]{baza}
					\caption{Relacijska shema baze podataka}
					
				\end{figure}
			
			\eject
			
			
		\section{Dijagram razreda}
		
			\textit{Potrebno je priložiti dijagram razreda s pripadajućim opisom. Zbog preglednosti je moguće dijagram razlomiti na više njih, ali moraju biti grupirani prema sličnim razinama apstrakcije i srodnim funkcionalnostima.}\\
			
			
			\textbf{\textit{dio 1. revizije}}\\
			
			
			\textit{Prilikom prve predaje projekta, potrebno je priložiti potpuno razrađen dijagram razreda vezan uz \textbf{generičku funkcionalnost} sustava. Ostale funkcionalnosti trebaju biti idejno razrađene u dijagramu sa sljedećim komponentama: nazivi razreda, nazivi metoda i vrste pristupa metodama (npr. javni, zaštićeni), nazivi atributa razreda, veze i odnosi između razreda.}\\
			
			\graphicspath{ {./slike/} }
			\begin{figure}[H]
				
				\includegraphics[width=\textwidth,height=\textheight,keepaspectratio]{Controller Class Diagram}
				\caption{\newline Dijagram Razreda Controller }
				
			\end{figure}
			
				\begin{figure}[H]
				
				\includegraphics[width=\textwidth,height=\textheight,keepaspectratio]{DTOdijagram}
				\caption{\newline Dijagram DTO razreda }
				
			\end{figure}
			
			\textbf{\textit{dio 2. revizije}}\\		
				
			
			
		
			\textit{Prilikom druge predaje projekta dijagram razreda i opisi moraju odgovarati stvarnom stanju implementacije}
			
			
			
			
			\eject
			
		\section{Dijagram stanja}
			
			
			\textbf{\textit{dio 2. revizije}}\\
			
			\textit{Potrebno je priložiti dijagram stanja i opisati ga. Dovoljan je jedan dijagram stanja koji prikazuje \textbf{značajan dio funkcionalnosti} sustava. Na primjer, stanja korisničkog sučelja i tijek korištenja neke ključne funkcionalnosti jesu značajan dio sustava, a registracija i prijava nisu. }
			
			
			\eject 
		
		\section{Dijagram aktivnosti}
			
			\textbf{\textit{dio 2. revizije}}\\
			
			 \textit{Potrebno je priložiti dijagram aktivnosti s pripadajućim opisom. Dijagram aktivnosti treba prikazivati značajan dio sustava.}
			
			\eject
		\section{Dijagram komponenti}
		
			\textbf{\textit{dio 2. revizije}}\\
		
			 \textit{Potrebno je priložiti dijagram komponenti s pripadajućim opisom. Dijagram komponenti treba prikazivati strukturu cijele aplikacije.}