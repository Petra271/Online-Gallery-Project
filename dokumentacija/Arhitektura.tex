\chapter{Arhitektura i dizajn sustava}

		Arhitektura se može podijeliti na tri podsustava:
		
		\begin{itemize}
			\item Web poslužitelj
			\item Web aplikacija
			\item Baza podataka
		\end{itemize}
		
		\begin{figure}[H]
			
			\centering
			\includegraphics[scale=0.25,keepaspectratio]{arhitektura}
			\caption{Arhitektura sustava}
			
		\end{figure}
	
		Web preglednik je program koji korisniku omogućuje pregled web-stranica i multimedijalnih sadržaja vezanih uz njih. Njime gledamo trenutnu stranicu koja je pisana u kodu koji preglednik prevodi u nešto svakome razumljivo. Korištenjem preglednika šalju se zahtjevi web poslužitelju.
		
		Web poslužitelj omogućava komunikaciju klijenta s aplikacijom. Za tu komunikaciju najčešće se koristi HTTP protokol, a poslužitelj zahtjev napisan u tom protokolu šalje web aplikaciji.
		
		Web aplikacija obrađuje zahtjeve i ovisno o zahtjevu, komunicira s bazom podataka te korisniku vraća odgovor u obliku HTML dokumenta koji web preglednik prevodi.
		
		Za izradu web stranice odlučili smo se za korištenje PostgreSQL baze podataka s kojom ćemo komunicirati putem Klijent-Server modela.
		
		Za frontend tj. klijentski  dio aplikacije koristit ćemo Vue.js. Vue.js je open-source JavaScript framework za izradu web aplikacija. On je dizajniran s ciljem da komplicirane aplikacije podijelimo na manje dijelove koje možemo ponovno koristiti. Razvojno okruženje za Vue.js je Microsoft Visual Studio.
		
		Za backend tj. serverski dio aplikacije koristit ćemo programski jezik Javu i Spring Boot framework. Radno okruženje za backend je IntelliJ IDEA.
		\newline
		Arhitektura Spring Boot framework-a koristi više slojeva:
		
		\begin{itemize}
			\item Sloj korisničke strane - implementiran u Vue.js
			\item Controller - povezuje korisnika i poslužitelja
			\item Service - obavlja poslovnu logiku i izračune
			\item Domain - implementira model podataka domene
			\item DAO (data access object) - obavlja sve operacije nad bazom podataka 
			\item Sloj baze podataka - obavlja pohranu podataka u neku bazu
		\end{itemize}

		

				
		\section{Baza podataka}
			
			Za našu web stranicu koristit ćemo relacijsku bazu podataka PostgreSQL. Baza podataka se sastoji od tablica, odnosno relacija koje su definirane svojim imenom i skupom atributa. Zadaća baze podataka je brza i jednostavna pohrana, izmjena i dohvat podataka za daljnju obradu. \newline Baza podataka ove aplikacije sastoji se od sljedećih tablica:
			\begin{itemize}
				\item Korisnik
				\item Kolekcija
				\item Komentar
				\item Transakcija
				\item KolekcijaNatjecaj
				\item KolekcijaIzlozba
				\item PosjetiteljIzlozba
				\item Djelo
				\item Izlozba
				\item Natjecaj
				
			\end{itemize}
		
			\subsection{Opis tablica}
				
				{\textbf{Korisnik} Ovaj entitet sadržava sve informacije o korisniku stranice. Sadrži atribute: Id, korisničko ime, ime, prezime, e-mail, lozinka, e-mail adresa PayPal računa, razina ovlasti korisnika te njegov portfolio u PDF-u. Ovaj entitet je u vezi \textit{Many-to-One} s entitetom Komentar preko atributa Id korisnika, u vezi \textit{Many-to-Many} s entitetom PosjetiteljIzlozba preko atributa Id korisnika, u vezi \textit{One-to-Many} s entitetom Transakcija u ulozi primatelja i pošiljatelja preko atributa	Id korisnika i u vezi \textit{One-to-Many} s entitetom Kolekcija preko atributa Id korisnika.}
				
				\begin{longtabu} to \textwidth {|X[10, l]|X[6, l]|X[14, l]|}
					
					\hline \multicolumn{3}{|c|}{\textbf{Korisnik}}	 \\[3pt] \hline
					\endfirsthead
					
					\hline \multicolumn{3}{|c|}{\textbf{Korisnik}}	 \\[3pt] \hline
					\endhead
					
					\hline 
					\endlastfoot
					
					\cellcolor{LightGreen}Id (PK) & INT&  Primarni ključ za korisnika	\\ \hline
					KorisnickoIme& VARCHAR &  Korisničko ime korisnika, ujedno i sekundarni ključ jer je jedinstveno za svakog korisnika\\ \hline 
					Ime	& VARCHAR &   Ime korisnika\\ \hline 
					Prezime & VARCHAR&   Prezime korisnika		\\ \hline 
					Email & VARCHAR &  E-mail korisnika, ujedno i sekundarni ključ jer je e-mail jedinstven za svakog korisnika  \\ \hline 
					Lozinka & VARCHAR&  Lozinka korisnika	\\ \hline 
					PayPalEmail & VARCHAR&  Lozinka korisnika	\\ \hline 
					RazinaOvlasti & VARCHAR&  Oznaka razine ovlasti korisnika	\\ \hline 
					PdfPortfolio & BYTEA&  Portfolio umjetnika u obliku pdfa, samo za umjetnike	\\ \hline 
					
					
				\end{longtabu}
			
				{\noindent\textbf{Kolekcija} Ovaj entitet sadržava sve informacije o kolekcijama na stranici. Sadrži atribute: Id kolekcije, naziv kolekcije, opis kolekcije, stil kolekcije te Id umjetnika čija je kolekcija. Ovaj entitet je u vezi \textit{One-to-Many} s entitetom KolekcijaNatjecaj preko atributa Id kolekcije, u vezi \textit{One-to-Many} s entitetom KolekcijaIzlozba preko atributa Id kolekcije, u vezi \textit{One-to-Many} s entitetom Djelo preko atributa Id kolekcije i u vezi \textit{Many-to-One} s entitetom Korisnik preko atributa Id umjetnika.}

				
				\begin{longtabu} to \textwidth {|X[10, l]|X[6, l]|X[14, l]|}
					
					\hline \multicolumn{3}{|c|}{\textbf{Kolekcija}}	 \\[3pt] \hline
					\endfirsthead
					
					\hline \multicolumn{3}{|c|}{\textbf{Kolekcija}}	 \\[3pt] \hline
					\endhead
					
					\hline 
					\endlastfoot
					
					\cellcolor{LightGreen}Id(PK) & INT&   Primarni ključ za kolekciju	\\ \hline
					Naziv	& VARCHAR &   Naziv kolekcije	\\ \hline 
					Opis & VARCHAR &  Opis kolekcije  \\ \hline 
					Stil & VARCHAR&  Stil kolekcije	\\ \hline 
					\cellcolor{LightBlue} IdUmjetnik (FK)& INT &  Strani ključ vezan uz korisnika s ulogom umjetnik	\\ \hline 
					
					
				\end{longtabu}

				{\noindent\textbf{Komentar} Ovaj entitet sadržava sve informacije o komentarima djela. Sadrži atribute: Id komentara, tekst komentara, vrijeme komentara, Id djela koje je komentirano napisan te Id autora koji je to djelo komentirao. Ovaj entitet je u vezi  \textit{Many-to-One} s entitetom Djelo preko atributa Id djelo, u vezi \textit{Many-to-One} s entitetom Korisnik preko atributa Id autor.}

				\begin{longtabu} to \textwidth {|X[10, l]|X[6, l]|X[14, l]|}
					
					\hline \multicolumn{3}{|c|}{\textbf{Komentar}}	 \\[3pt] \hline
					\endfirsthead
					
					\hline \multicolumn{3}{|c|}{\textbf{Komentar}}	 \\[3pt] \hline
					\endhead
					
					\hline 
					\endlastfoot
					
					\cellcolor{LightGreen}Id (PK)& INT&  Primarni ključ komentara	\\ \hline
					Tekst	& VARCHAR &   Tekst komentara	\\ \hline 
					Vrijeme & TIMESTSAMP &  Vrijeme kada je korisnik ostavio komentar  \\ \hline 
					\cellcolor{LightBlue}IdDjelo(FK) & INT&  Strani ključ vezan uz djelo	\\ \hline 
					\cellcolor{LightBlue} IdAutor(FK)& INT &  Strani ključ vezan uz korisnika koji je komentirao	\\ \hline 
					
					
				\end{longtabu}

				{\noindent\textbf{Transakcija} Ovaj entitet sadržava sve informacije o pojedinoj  transakciji kupnje djela. Sadrži atribute: Id transakcije, iznos, vrijeme transakcije, Id djela koje je kupljeno, Id platitelja koji je kupio djelo te Id korisnika koji prima novce. Ovaj entitet je u vezi \textit{Many-to-One} s entitetom Korisnik
			 	preko atributa Id platitelj i Id primatelj i u vezi \textit{One-to-One} s entitetom Djelo preko atributa Id djelo.}


				\begin{longtabu} to \textwidth {|X[10, l]|X[6, l]|X[14, l]|}
					
					\hline \multicolumn{3}{|c|}{\textbf{Transakcija}}	 \\[3pt] \hline
					\endfirsthead
					
					\hline \multicolumn{3}{|c|}{\textbf{Transakcija}}	 \\[3pt] \hline
					\endhead
					
					\hline 
					\endlastfoot
					
					\cellcolor{LightGreen}Id(PK) & INT& Primarni ključ transakcije  	\\ \hline
					Iznos	& FLOAT &   	Iznos plaćene cijene u kunama\\ \hline 
					Vrijeme & TIMESTAMP &   Vrijeme kada je transakcija tj. uplata sredstava završena\\ \hline 
					\cellcolor{LightBlue}IdDjelo(FK) & INT	&  Strani ključ vezan uz kupljeno djelo	\\ \hline 
					\cellcolor{LightBlue}IdPlatitelj(FK) & INT	&  Strani ključ vezan uz korisnika koji je uplatio sredstva\\ \hline 
					\cellcolor{LightBlue} IdPrimatelj(FK)& INT &   Strani ključ vezan uz korisnika koji prima novce	\\ \hline 
					
					
				\end{longtabu}

				{\noindent\textbf{KolekcijaNatjecaj} Ovaj entitet sadržava sve informacije o odnosu kolekcije i natjecaja. Sadrži atribute: Id kolekcije te Id natječaja u kojem je kolekcija. Ovaj entitet je u vezi \textit{Many-to-One} s entitetom Kolekcija preko atributa Id kolekcija i u vezi \textit{Many-to-One} s entitetom Natjecaj preko atributa Id natjecaj.}


				\begin{longtabu} to \textwidth {|X[10, l]|X[6, l]|X[14, l]|}
					
					\hline \multicolumn{3}{|c|}{\textbf{KolekcijaNatjecaj}}	 \\[3pt] \hline
					\endfirsthead
					
					\hline \multicolumn{3}{|c|}{\textbf{KolekcijaNatjecaj}}	 \\[3pt] \hline
					\endhead
					
					\hline 
					\endlastfoot
					
					 IdKolekcija(PK,FK)	& INT &   Dio primarnog ključa vezan uz kolekciju, ujedno i  strani ključ	\\ \hline 
					 IdNatjecaj(PK,FK)	& INT &  Dio primarnog ključa vezan uz natječaj, ujedno i  strani ključ	\\ \hline 
					
					
				\end{longtabu}
			
				{\noindent\textbf{KolekcijaIzlozba} Ovaj entitet sadržava sve informacije o odnosu kolekcije i izložbe. Sadrži atribute: Id izložbe te Id kolekcije u kojem je izložba. Ovaj entitet je u vezi \textit{Many-to-One} s entitetom Kolekcija preko atributa Id kolekcija i u vezi \textit{Many-to-One} s entitetom Izlozba preko atributa Id izlozba.}


				\begin{longtabu} to \textwidth {|X[10, l]|X[6, l]|X[14, l]|}
					
					\hline \multicolumn{3}{|c|}{\textbf{KolekcijaIzlozba}}	 \\[3pt] \hline
					\endfirsthead
					
					\hline \multicolumn{3}{|c|}{\textbf{KolekcijaIzlozba}}	 \\[3pt] \hline
					\endhead
					
					\hline 
					\endlastfoot
					
					 IdKolekcija(PK,FK)	& INT &   Dio primarnog ključa vezan uz kolekciju, ujedno i  strani ključ	\\ \hline 
					 IdIzlozba(PK,FK)	& INT &   Dio primarnog ključa vezan uz izložbu, ujedno i  strani ključ	\\ \hline 
					
					
				\end{longtabu}

				{\noindent\textbf{PosjetiteljIzlozba} Ovaj entitet sadržava sve informacije o odnosu posjetitelja i izložbe. Sadrži atribute: Id posjetitelja te Id izložbe koju posjetitelj pregledava. Ovaj entitet je u vezi \textit{Many-to-One} s entitetom Posjetitelj preko atributa Id posjetitelj i u vezi \textit{Many-to-One} s entitetom Izlozba preko atributa Id izlozba.}

			
				\begin{longtabu} to \textwidth {|X[10, l]|X[6, l]|X[14, l]|}
					
					\hline \multicolumn{3}{|c|}{\textbf{PosjetiteljIzlozba}}	 \\[3pt] \hline
					\endfirsthead
					
					\hline \multicolumn{3}{|c|}{\textbf{PosjetiteljIzlozba}}	 \\[3pt] \hline
					\endhead
					
					\hline 
					\endlastfoot
					
					 IdIzlozba(PK,FK)	& INT &   Dio primarnog ključa vezan uz izložbu,ujedno i  strani ključ	\\ \hline 
					 IdPosjetitelj(PK,FK)& INT &  Dio primarnog ključa vezan uz korisnika koji je posjetio izložbu, ujedno i strani ključ	\\ \hline 
					
					
				\end{longtabu}

				{\noindent\textbf{Djelo} Ovaj entitet sadržava sve informacije o pojedinom djelu. Sadrži atribute: Id djela, naziv, opis, cijena, stil, djelo u binarnom formatu te Id kolekcije u kojem je djelo. Ovaj entitet je u vezi \textit{Many-to-One} s entitetom Kolekcija preko atributa Id kolekcija, u vezi \textit{One-to-One} s entitetom Transakcija preko atributa Id djelo.}


				\begin{longtabu} to \textwidth {|X[10, l]|X[6, l]|X[14, l]|}
					
					\hline \multicolumn{3}{|c|}{\textbf{Djelo}}	 \\[3pt] \hline
					\endfirsthead
					
					\hline \multicolumn{3}{|c|}{\textbf{Djelo}}	 \\[3pt] \hline
					\endhead
					
					\hline 
					\endlastfoot
					
					\cellcolor{LightGreen}Id(PK) & INT& Primarni ključ djela 	\\ \hline
					Naziv	& VARCHAR&   Naziv djela\\ \hline 
					Opis	& VARCHAR &   Autorov tj. umjetnikov opis djela\\ \hline 
					Cijena	& FLOAT &   	Cijena djela u kunama\\ \hline 
					Stil & VARCHAR & Stil djela\\ \hline 
					blob & BYTEA & Djelo u pdf formatu \\ \hline 
					\cellcolor{LightBlue} IdKolekcija(FK)& INT &   Strani ključ vezan uz kolekciju u kojem je djelo	\\ \hline 
					
					
				\end{longtabu}

				{\noindent\textbf{Izlozba} Ovaj entitet sadržava sve informacije o  izložbama. Sadrži atribute: Id izložbe, naziv, opis, vrijeme početka izložbe, vrijeme trajanja, provizija prilikom kupnje djela, stil te Id natječaja unutar kojeg je izloženo djelo. Ovaj entitet je u vezi \textit{One-to-Many} s entitetom PosjetiteljIzlozba preko atributa Id izlozba, u vezi \textit{One-to-Many} s entitetom KolekcijaIzlozba preko atributa Id izlozba i u vezi \textit{Many-to-One} s entitetom Natjecaj preko atributa Id natjecaj.}

				
				\begin{longtabu} to \textwidth {|X[10, l]|X[6, l]|X[14, l]|}
					
					\hline \multicolumn{3}{|c|}{\textbf{Izlozba}}	 \\[3pt] \hline
					\endfirsthead
					
					\hline \multicolumn{3}{|c|}{\textbf{Izlozba}}	 \\[3pt] \hline
					\endhead
					
					\hline 
					\endlastfoot
					
					\cellcolor{LightGreen}Id(PK) & INT& Primarni ključ izložbe 	\\ \hline
					Naziv	& VARCHAR&   Naziv djela\\ \hline 
					Opis	& VARCHAR &   Autorov tj. umjetnikov opis djela\\ \hline 
					VrijemePocetka& TIMESTAMP &   	Vrijeme u kojem počinje izložba\\ \hline 
					VrijemeTrajanja& INTERVAL &   	Interval u kojem izložba traje\\ \hline 
					Provizija &FLOAT & Broj izražen u postocima o kojem ovisi koliko stranica uzima proviziju od cijene djela
 tj. koliko će od ukupne cijene djela novaca dobiti korisnik. Korisnik dobiva (100-Provizija)/100*CijenaDjela kuna dok ostatak novaca ide stranici.\\ \hline 
					Stil & VARCHAR & Stil izložbe\\ \hline 
					\cellcolor{LightBlue} IdNatjecaj(FK)& INT &   Strani ključ vezan uz natječaj u kojem je djelo	\\ \hline 
					
					
				\end{longtabu}

				{\noindent\textbf{Natjecaj} Ovaj entitet sadržava sve informacije o natječajima. Sadrži atribute: Id natječaja, naziv, opis, vrijeme početka natječaja, vrijeme trajanja, provizija prilikom kupnje djela te stil natječaja. Ovaj entitet je u vezi \textit{One-to-One} s entitetom Izlozba preko atributa Id natjecaj i u vezi \textit{One-to-Many} s entitetom KolekcijaNatjecaj preko atributa Id natjecaj.}


				\begin{longtabu} to \textwidth {|X[10, l]|X[6, l]|X[14, l]|}
					
					\hline \multicolumn{3}{|c|}{\textbf{Natjecaj}}	 \\[3pt] \hline
					\endfirsthead
					
					\hline \multicolumn{3}{|c|}{\textbf{Natjecaj}}	 \\[3pt] \hline
					\endhead
					
					\hline 
					\endlastfoot
					
					\cellcolor{LightGreen}Id(PK) & INT& Primarni ključ natječaja 	\\ \hline
					Naziv	& VARCHAR&   Naziv natječaja\\ \hline 
					Opis	& VARCHAR &   Administratorov opis natječaja\\ \hline 
					VrijemePocetka& TIMESTAMP &   	Vrijeme u kojem počinje natječaj\\ \hline 
					VrijemeTrajanja& INTERVAL &   	Interval u kojem natječaj traje\\ \hline 
					Provizija &FLOAT & Broj izražen u postocima o kojem ovisi koliko stranica uzima proviziju od cijene djela
 tj. koliko će od ukupne cijene djela novaca dobiti korisnik. Korisnik dobiva (100-Provizija)/100*CijenaDjela kuna dok ostatak novaca ide stranici.\\ \hline 
					Stil & VARCHAR & Stil natječaja\\ \hline 
					
					
				\end{longtabu}

			
			\subsection{Dijagram baze podataka}
				\begin{figure}[H]
					
					\includegraphics[width=\textwidth,height=\textheight,keepaspectratio]{baza}
					\caption{Relacijska shema baze podataka}
					
				\end{figure}
			
			\eject
			
			
		\section{Dijagram razreda}
		
			{Na slikama 4.2 i 4.3 prikazani su dijagrami razreda u backend dijelu arhitekture.
			Razredi na slici 4.2 nasljeduju razred Controller. Metode tih razreda manipuliraju
			DTO-ima (Data transfer object) koji su dohvaceni pomoću metoda implementiranih
			u Model razredima. Metode implementirane u Controller razredima vracaju JSON
			datoteke s HTTP statusnim kodom. Slika 4.3 prikazuje dijagram DTO razreda. 	
			\vspace{10mm}
			
			\graphicspath{ {./slike/} }
			\begin{figure}[H]
				
				\includegraphics[width=\textwidth,height=\textheight,keepaspectratio]{Controller Class Diagram}
				\caption{\newline Dijagram Razreda Controller }
				
			\end{figure}
			
				\begin{figure}[H]
				
				\includegraphics[width=\textwidth,height=\textheight,keepaspectratio]{DTOdijagram}
				\caption{\newline Dijagram DTO razreda }
				
			\end{figure}
			
			\textbf{\textit{dio 2. revizije}}\\		
				
			
			
		
			\textit{Prilikom druge predaje projekta dijagram razreda i opisi moraju odgovarati stvarnom stanju implementacije}
			
			
			
			
			\eject
			
		\section{Dijagram stanja}
			
			
			\textbf{\textit{dio 2. revizije}}\\
			
			\textit{Potrebno je priložiti dijagram stanja i opisati ga. Dovoljan je jedan dijagram stanja koji prikazuje \textbf{značajan dio funkcionalnosti} sustava. Na primjer, stanja korisničkog sučelja i tijek korištenja neke ključne funkcionalnosti jesu značajan dio sustava, a registracija i prijava nisu. }
			
			
			\eject 
		
		\section{Dijagram aktivnosti}
			
			\textbf{\textit{dio 2. revizije}}\\
			
			 \textit{Potrebno je priložiti dijagram aktivnosti s pripadajućim opisom. Dijagram aktivnosti treba prikazivati značajan dio sustava.}
			
			\eject
		\section{Dijagram komponenti}
		
			\textbf{\textit{dio 2. revizije}}\\
		
			 \textit{Potrebno je priložiti dijagram komponenti s pripadajućim opisom. Dijagram komponenti treba prikazivati strukturu cijele aplikacije.}